

\setlength{\absparsep}{18pt} %espaçamento dos parágrafos do resumo
	\begin{resumo}
	A busca por uma maior segurança e privacidade em diversas áreas da comunicação humana
foi um grande ponto de partida para o surgimento da Criptografia, onde inicialmente surgiu a Criptografia Simétrica e após um tempo
a Criptografia Assimétrica. Surgiram vários tipos de algoritmos para esses dois tipos de criptografia, como por exemplo o Algoritmo AES (Criptografia Simétrica)  e Algoritmo RSA (Criptografia Assimétrica).


Nesse trabalho será feito uma explicação sobre o que é a criptografia, os seus tipos e sobre o algoritmo padrão de parâmetro AES
utilizado nesse projeto para a criptografia de textos. Após isso, esse algoritmo será implementado no software
computacional MATLAB. Com essa implementação será possível analisar o desempenho desse algoritmo.
	
		\noindent
		\textbf{Palavras-chaves}: Grafos, Problema dos Menores Caminhos, Floyd-Warshall, Dijkstra, Iterações
	\end{resumo}
